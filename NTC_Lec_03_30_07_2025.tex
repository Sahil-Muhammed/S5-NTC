\documentclass[hidelinks]{article}

% Page setup
\usepackage[a4paper, total={7in, 10in}]{geometry}
\usepackage{fancyhdr}
\usepackage{setspace}
\setlength{\parskip}{0.5em}
\linespread{1.2}

% Math & theorem environments
\usepackage{amsmath, amssymb, amsthm}
\usepackage{enumitem}

% TikZ & colors
\usepackage{tikz}
\usetikzlibrary{angles,calc,decorations.pathreplacing}
\usepackage{tkz-euclide}
\usepackage[dvipsnames]{xcolor}

% Hyperref
\usepackage[unicode]{hyperref}
\usepackage[all]{hypcap}

% Colors
\definecolor{carminered}{rgb}{1.0, 0.0, 0.22}
\definecolor{capri}{rgb}{0.0, 0.75, 1.0}
\definecolor{brightlavender}{rgb}{0.75, 0.58, 0.89}

% Theorem styles
\newtheoremstyle{mystyle}%
  {6pt}   % space above
  {6pt}   % space below
  {\itshape} % body font
  {} % indent
  {\bfseries} % theorem head font
  {.} % punctuation after theorem head
  {0.5em} % space after theorem head
  {} % theorem head spec
\theoremstyle{mystyle}
\newtheorem{theorem}{Theorem}[section]
\newtheorem{definition}{Definition}[section]
\newtheorem{proposition}{Proposition}[section]

% Header & footer
\pagestyle{fancy}
\fancyhf{}
\fancyhead[L]{Number Theory \& Cryptography}
\fancyhead[R]{\thepage}
\fancyfoot[C]{S. Muhammed -- 2025}

% Title
\title{\textbf{\LARGE Number Theory \& Cryptography}}
\author{S. Muhammed}
\date{July 30, 2025}

\begin{document}
\maketitle

\section{Fundamental Theorem of Arithmetic}

\begin{theorem}[Fundamental Theorem of Arithmetic]
Let $n \in \mathbb{Z}$. Then
\[
n = (\pm) p_1^{\alpha_1} \cdot p_2^{\alpha_2} \cdot \dots \cdot p_r^{\alpha_r},
\]
where $p_i$ are primes and $\alpha_i \in \mathbb{N} \cup \{0\}$.
\end{theorem}

\textbf{Examples:}
\[
24 = 2^3 \cdot 3^1 \cdot 5^0, \qquad 30 = 2^1 \cdot 3^1 \cdot 5^1
\]

\subsection{gcd and lcm via prime factorization}
\[
\gcd(a, b) = p_1^{\min(\alpha_1, \beta_1)} \cdot p_2^{\min(\alpha_2, \beta_2)} \cdots p_r^{\min(\alpha_r, \beta_r)}
\]
\[
\mathrm{lcm}(a, b) = p_1^{\max(\alpha_1, \beta_1)} \cdot p_2^{\max(\alpha_2, \beta_2)} \cdots p_r^{\max(\alpha_r, \beta_r)}
\]

\textbf{Example:}
\[
\gcd(24, 30) = 2^{\min(3,1)} \cdot 3^{\min(1,1)} \cdot 5^{\min(0,1)} = 2^1 \cdot 3^1 \cdot 5^0 = 6
\]

\subsection{Remark}
The \textbf{Integer Factorization Problem (IFP)} asks for the factorization of a given integer $n$ into primes.  
For example, given $n$, find $p,q$ such that $n=pq$.  
RSA encryption (a widely used public-key cryptosystem) would be broken if the IFP were solved efficiently.

\section{Euclidean Algorithm}

Let $a, b \in \mathbb{Z}$ with $a > b$. Then there exist $q, r \in \mathbb{Z}$ such that
\[
a = bq + r, \qquad 0 \leq r < b.
\]

\subsection{Steps}
\begin{align*}
a &= bq_1 + r_1, & 0 \le r_1 < b \\
b &= r_1 q_2 + r_2, & 0 \le r_2 < r_1 \\
r_1 &= r_2 q_3 + r_3, & 0 \le r_3 < r_2 \\
r_2 &= r_3 q_4 + r_4, & 0 \le r_4 < r_3 \\
&\vdots \\
r_{j-2} &= r_{j-1} q_j + r_j, & 0 \le r_j < r_{j-1} \\
r_{j-1} &= r_j q_{j+1} \\
\end{align*}

Then $r_j = \gcd(a, b)$.

\subsection{Example: $\gcd(36, 21)$}
\begin{align*}
36 &= 21 \cdot 1 + 15 \\
21 &= 15 \cdot 1 + 6 \\
15 &= 6 \cdot 2 + 3 \\
6  &= 3 \cdot 2 + 0
\end{align*}

Hence $\gcd(36, 21) = 3$.

\subsection{Recursive Implementation in C}
\begin{verbatim}
int gcd(int a, int b) {
    if (a < b) return gcd(b, a);
    if (b == 0) return a;
    return gcd(b, a % b);
}
\end{verbatim}

\[
\gcd(36, 21) = \gcd(21, 15) = \gcd(15, 6) = \gcd(6, 3) = \gcd(3, 0) = 3
\]

\section{Recurrence Relation}

\[
T(n) =
\begin{cases}
  3 & \text{if } n = 0, \\
  T(n-1) + 3 & \text{if } n \geq 1.
\end{cases}
\]

By repeated substitution:
\[
T(n) = T(n-1) + 2 = T(n-2) + 4 = \dots = 2 + 2n.
\]

\end{document}

