\documentclass[hidelinks]{article}

% Page setup
\usepackage[a4paper, total={7in, 10in}]{geometry}
\usepackage{fancyhdr}
\usepackage{setspace}
\setlength{\parskip}{0.5em}
\linespread{1.1}

% Math & theorem environments
\usepackage{amsmath, amssymb, amsthm}
\usepackage{enumitem}

% TikZ & colors
\usepackage{tikz}
\usetikzlibrary{angles,calc,decorations.pathreplacing}
\usepackage{tkz-euclide}
\usepackage{circuitikz}
\usepackage[dvipsnames]{xcolor}

% Hyperref
\usepackage[unicode]{hyperref}
\usepackage[all]{hypcap}

% Colors
\definecolor{carminered}{rgb}{1.0, 0.0, 0.22}
\definecolor{capri}{rgb}{0.0, 0.75, 1.0}
\definecolor{brightlavender}{rgb}{0.75, 0.58, 0.89}

% Theorem styles
\newtheoremstyle{mystyle}%
  {6pt}   % space above
  {6pt}   % space below
  {\itshape} % body font
  {} % indent
  {\bfseries} % theorem head font
  {.} % punctuation after theorem head
  {0.5em} % space after theorem head
  {} % theorem head spec
\theoremstyle{mystyle}
\newtheorem{definition}{Definition}[section]
\newtheorem{theorem}{Theorem}[section]
\newtheorem{proposition}{Proposition}[section]

% Header & footer
\pagestyle{fancy}
\fancyhf{}
\fancyhead[L]{Number Theory \& Cryptography}
\fancyhead[R]{\thepage}
\fancyfoot[C]{S. Muhammed -- 2025}

% Title
\title{\textbf{\LARGE Number Theory \& Cryptography}}
\author{S. Muhammed}
\date{July 31, 2025}

\begin{document}
\maketitle

\section{Extended Euclid's Algorithm}

\begin{itemize}
    \item Returns the values of $a, b, d$ satisfying $ax + by = d$ given integers $x, y$.
    \item It is \emph{extended} because it returns both the gcd and coefficients of the above equation.
\end{itemize}

\begin{verbatim}
ExtendedEuclid(x, y)
if (y == 0) return (x, 1, 0)
else{
        q = x / y;
        r = x % y;
        (d, a, b) = ExtendedEuclid(y, r);
        return (d, b, a - bq);
}
\end{verbatim}


\subsection*{Example: 35 and 20}

\begin{figure}[!ht]
\centering
\resizebox{0.9\textwidth}{!}{%
\begin{circuitikz}
\tikzstyle{every node}=[font=\LARGE]

% Your original TikZ nodes go here directly
\draw  (1.5,16.25) rectangle (6,8.5);
\draw  (7.75,14) rectangle (12.25,6.25);
\draw  (14.25,12) rectangle (18.75,4.25);
\draw  (20,5.75) rectangle (24,2.5);
\node [font=\LARGE] at (2.5,16.75) {$x$};
\node [font=\LARGE] at (5,16.75) {$y$};
\node [font=\LARGE] at (2.5,15.75) {35};
\node [font=\LARGE] at (5,15.75) {20};
\node [font=\LARGE] at (8.5,14.5) {$x$};
\node [font=\LARGE] at (11.5,14.5) {$y$};
\node [font=\LARGE] at (15.25,12.5) {$x$};
\node [font=\LARGE] at (17.75,12.5) {$y$};
\node [font=\LARGE] at (20.75,6.25) {$x$};
\node [font=\LARGE] at (23.25,6.25) {$y$};
\node [font=\LARGE] at (2.5,15.75) {35};
\node [font=\LARGE] at (5,15.75) {20};
\node [font=\LARGE] at (8.5,13.5) {20};
\node [font=\LARGE] at (11.5,13.5) {15};
\node [font=\LARGE] at (15.25,11.5) {15};
\node [font=\LARGE] at (17.75,11.5) {5};
\node [font=\LARGE] at (20.75,5.25) {5};
\node [font=\LARGE] at (23.25,5.25) {0};
\node [font=\LARGE] at (3.75,14.5) {$q = 1$};
\node [font=\LARGE] at (3.75,13.5) {$r = 15$};
\node [font=\LARGE] at (10,11.75) {$q = 1$};
\node [font=\LARGE] at (10,10.75) {$r = 5$};
\node [font=\LARGE] at (16.5,10) {$q = 3$};
\node [font=\LARGE] at (16.5,9) {$r = 0$};
\node [font=\LARGE] at (3.75,12.5) {$(d, a, b)$};
\node [font=\LARGE] at (10,9.75) {$(d, a, b)$};
\node [font=\LARGE] at (16.5,8) {$(d, a, b)$};
\draw [->, >=Stealth] (6,13.5) -- (7.75,13.5);
\draw [->, >=Stealth] (12.25,11.75) -- (14.25,11.75);
\draw [->, >=Stealth] (18.75,5.5) -- (20,5.5);
\node [font=\LARGE] at (22,4) {$(5, 1, 0)$};
\node [font=\LARGE] at (21.5,4) {$$};
\node [font=\LARGE] at (21.5,4) {$$};
\draw [->, >=Stealth] (20,4.25) -- (17.75,6.75);
\node [font=\LARGE] at (16.5,5) {$(5, 0, 1)$};
\node [font=\LARGE] at (10,8) {$(d, b, a-bq)$};
\node [font=\LARGE] at (3.75,10.5) {$(d, b, a-bq)$};
\node [font=\LARGE] at (16.5,7) {$(5, 1, 0)$};
\node [font=\LARGE] at (10,8.75) {(5, 0, 1)};
\node [font=\LARGE] at (3.75,11.5) {(5, 1, -1)};
\node [font=\LARGE] at (16.5,6) {$(d, b, a-bq)$};
\node [font=\LARGE] at (10,7) {(5, 1, -1)};
\node [font=\LARGE] at (3.75,9.5) {(5, -1, 2)};
\draw [->, >=Stealth] (14.25,5) -- (11.25,8.75);
\draw [->, >=Stealth] (7.75,7) -- (5,11.5);
\draw [ color={rgb,255:red,51; green,209; blue,122} , dashed] (3.75,9.5) ellipse (1.75cm and 0.5cm);
% ... (trimmed, keep the rest of your drawing here) ...
\end{circuitikz}
}
\caption{Extended Euclidean Algorithm; $5 = (-1)35 + (2)20$}
\end{figure}

\section{Recurrence Relation}

\[
T(n) =
\begin{cases}
	2 & \text{if } n = 0, \\
	2 + T(n-1) & \text{if } n > 0.
\end{cases}
\]

\begin{figure}[!ht]
\centering
\resizebox{0.25\textwidth}{!}{%
\begin{circuitikz}
\tikzstyle{every node}=[font=\LARGE]
\node [font=\LARGE] at (6.5,15) {a};
\node [font=\LARGE] at (3.75,13.25) {b $<$ a/2};
\node [font=\LARGE] at (6.5,13.25) {a/2};
\node [font=\LARGE] at (9.5,13.25) {c $>$ a/2};
\draw (2,14.25) to[short] (11.25,14.25);
\draw (2,14.5) to[short] (2,13.75);
\draw (11.25,13.75) to[short] (11.25,14.5);
\draw (6.5,14.5) to[short] (6.5,13.75);
\end{circuitikz}
}%

\label{fig:my_label}
\end{figure}

\subsection*{Examples:}
\begin{itemize}
    \item $\gcd(8, 3) = \gcd(3, 2) = \gcd(2, 1) = \gcd(1, 0)$
    \item $\gcd(8, 5) = \gcd(5, 3) = \gcd(3, 2) = \gcd(2, 1) = \gcd(1, 0)$
\end{itemize}

Thus, complexity: $\Theta(\log_2 a)$.
\section{Congruence}
\begin{definition}
For integers $a, b$, and $n > 0$:
\[
a \equiv b \pmod{n} \quad \iff \quad n \mid (a - b).
\]
\end{definition}

Examples:
\[
5 \equiv 2 \pmod{3}, \quad 5 \not\equiv 3 \pmod{3}.
\]

\subsection*{Remainder Classes mod 5}
\[
\begin{aligned}
[0] &= \{\dots, -10, -5, 0, 5, 10, \dots\}, \\
[1] &= \{\dots, -9, -4, 1, 6, 11, \dots\}, \\
[2] &= \{\dots, -8, -3, 2, 7, 12, \dots\}, \\
[3] &= \{\dots, -7, -2, 3, 8, 13, \dots\}, \\
[4] &= \{\dots, -6, -1, 4, 9, 14, \dots\}.
\end{aligned}
\]

Thus,
\[
\mathbb{Z} = [0] \cup [1] \cup [2] \cup [3] \cup [4].
\]

\subsection*{Properties}
\begin{enumerate}[label=(\arabic*)]
    \item $(a \pm b) \bmod n = (a \bmod n) \pm (b \bmod n)$
    \item $(a \cdot b) \bmod n = (a \bmod n)(b \bmod n)$
    \item $a \equiv b \pmod{n} \;\Rightarrow\; a \bmod n = b \bmod n$
\end{enumerate}

\subsection*{Theorems}
\begin{enumerate}[label=(\roman*)]
    \item $a \equiv b \pmod{n} \;\Rightarrow\; b \equiv a \pmod{n}, \quad (a-b) \equiv 0 \pmod{n}$.
    \item $a \equiv b \pmod{n},\; b \equiv c \pmod{n} \;\Rightarrow\; a \equiv c \pmod{n}$.
    \item $a \equiv b \pmod{n},\; c \equiv d \pmod{n} \;\Rightarrow\; a \pm b \equiv c \pm d \pmod{n}$.
\end{enumerate}

\end{document}

