\documentclass[hidelinks]{article}

% Page setup
\usepackage[a4paper, total={7in, 10in}]{geometry}
\usepackage{fancyhdr}
\usepackage{setspace}
\setlength{\parskip}{0.5em}
\linespread{1.2}

% Math & theorem environments
\usepackage{amsmath, amssymb, amsthm}
\usepackage{enumitem}

% TikZ & colors
\usepackage{tikz}
\usetikzlibrary{angles,calc,decorations.pathreplacing}
\usepackage{tkz-euclide}
\usepackage[dvipsnames]{xcolor}

% Hyperref
\usepackage[unicode]{hyperref}
\usepackage[all]{hypcap}

% Colors
\definecolor{orange}{RGB}{255, 165, 0}

% Theorem styles
\newtheoremstyle{mystyle}%
  {6pt}   % space above
  {6pt}   % space below
  {\itshape} % body font
  {} % indent
  {\bfseries} % theorem head font
  {.} % punctuation after theorem head
  {0.5em} % space after theorem head
  {} % theorem head spec
\theoremstyle{mystyle}
\newtheorem{definition}{Definition}[section]
\newtheorem{theorem}{Theorem}[section]
\newtheorem{proposition}{Proposition}[section]

% Header & footer
\pagestyle{fancy}
\fancyhf{}
\fancyhead[L]{Number Theory \& Cryptography}
\fancyhead[R]{\thepage}
\fancyfoot[C]{S. Muhammed -- 2025}

% Title
\title{\textbf{\LARGE Number Theory \& Cryptography}}
\author{S. Muhammed}
\date{July 24, 2025}

\begin{document}
\maketitle
\section{Number Theory}
\begin{itemize}
    \item For $n \in \mathbb{Z}$, $\pm 1, \pm n$ are trivial divisors of $n$.
    \item For $p \in \mathbb{N}$ with $p > 1$, $p$ is prime if it has only positive trivial divisors.
    \item Integers other than primes are called composite numbers (e.g. $n = 4 \cdot p$).
\end{itemize}

\begin{definition}[Division Theorem]
For $a, b \in \mathbb{Z}$ with $a > b$ and $b \neq 0$, there exist \textbf{unique} $q, r \in \mathbb{Z}$ such that
\[
a = bq + r, \qquad 0 \leq r < b.
\]
If $b \nmid a$, then $0 < r < b$.
\end{definition}

\section{Proof of Uniqueness in Division Theorem}
Assume $q_1 \neq q$, $r \neq r_1$, $b \neq 0$.  

For $a, b \in \mathbb{Z}$, $\exists$  $q, r \in \mathbb{Z}$ such that 
\[
a = bq + r \quad \longrightarrow (1)
\]
and also $\exists \, q_1, r_1 \in \mathbb{Z}$ such that 
\[
a = bq_1 + r_1 \quad \longrightarrow (2)
\]

Subtracting $(1)$ and $(2)$:
$$ bq_1 + r_1 = bq + r \Rightarrow b(q_1 - q) = (r - r_1)$$ 

Thus $b \mid (r-r_1)$, which contradicts the condition $0 \leq r, r_1 < b$ unless $r = r_1$.  
Hence $r = r_1$, and therefore
\[
q = \frac{a-r}{b}, \quad q_1 = \frac{a-r_1}{b} \Rightarrow q = q_1
\]
Thus both $q,r$ are unique.

\section{Greatest Common Divisor (GCD)}
\begin{definition}
For $a, b \in \mathbb{Z}$, $d \in \mathbb{Z}$ is called a \emph{common divisor} if $d \mid a$ and $d \mid b$.  
The greatest among all such divisors is called the \emph{greatest common divisor}, denoted $\gcd(a,b)$ or $(a,b)$.
\end{definition}

Examples:
\[
\gcd(10,5) = 5, \quad \gcd(5,0) = 5, \quad \gcd(5,2) = 1 \ (\text{co-prime}).
\]

\section{Least Common Multiple (LCM)}
\begin{itemize}
    \item $[10,20] = \mathrm{LCM}(10,20) = 20$.
    \item $a \cdot b = \gcd(a,b) \cdot \mathrm{lcm}(a,b)$.
    \item $a \cdot b = (a,b)[a,b]$.
\end{itemize}

\begin{theorem}[Bézout]
Let $a, b \in \mathbb{Z}$ and let $d = \gcd(a,b)$. Then there exist $x,y \in \mathbb{Z}$ such that
\[
d = ax + by
\]
\end{theorem}

\begin{proof}
\textcolor{orange}{Optional}

If $a=b=0$, then $\gcd(a,b)=0$, and the identity $0 = 0 \cdot x + 0 \cdot y$ holds trivially.  
Assume not both $a,b$ are zero, and set $d=\gcd(a,b)>0$.

Consider
\[
S = \{ax + by \mid x,y \in \mathbb{Z},\ ax+by>0\}.
\]
By the well-ordering principle, $S$ has a least element $m$.  
So there exist $x_0,y_0 \in \mathbb{Z}$ such that $m=ax_0+by_0$.

\textbf{(i) $m \mid a,b$.}  
Divide $a$ by $m$:
\[
a = qm + r, \quad 0 \leq r < m.
\]
But
\[
r = a - qm = a(1-qx_0) + b(-qy_0).
\]
Thus $r \in S$. By minimality, $r=0$, so $m \mid a$. Similarly, $m \mid b$.

\textbf{(ii) $m$ is greatest.}  
If $c$ is any common divisor of $a,b$, then $c \mid (ax+by)$ for all $x,y$, in particular $c \mid m$.  
Thus $m$ is divisible by all common divisors.

Therefore $m = \gcd(a,b) = d$, and $d = ax_0 + by_0$.

\end{proof}


\begin{theorem}
For $a,b,c \in \mathbb{Z}$, if $c \mid ab$ and $\gcd(a,c)=1$, then $c \mid b$.
\end{theorem}

\begin{proof}
Since $\gcd(a,c)=1$, $\exists \, x,y \in \mathbb{Z}$ such that $ax+cy=1$.  
Multiplying through by $b$:
\[
abx + cby = b.
\]

Since $c \mid ab$, $\exists$ $k \in \mathbb{Z}$ such that $ab = ck$. So,

$$ckx + cby = c(kx + by) = b \Rightarrow c \mid b$$

\end{proof}

\begin{theorem}
Let $p$ be prime. For $a,b \in \mathbb{Z}$, if $p \mid ab$, then $p \mid a$ or $p \mid b$.
\end{theorem}

\begin{proof}
\begin{itemize}
    \item If $p \nmid a$, then $\gcd(p,a)=1$, so $p \mid b$.
    \item If $p \nmid b$, then $\gcd(p,b)=1$, so $p \mid a$.
    \item If $p \mid a$ and/or $p \mid b$, the statement is already true.
\end{itemize}
\end{proof}

\end{document}
